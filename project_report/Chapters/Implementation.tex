\chapter{Implementation}

\section{Languages and Frameworks}
\subsection{Front end} 
Since the front end of the project is a web app , HTML and CSS were used to create the user interface. However for the programming Typescript was used instead of Javascript.

\paragraph{Typescript} is a super set of Javascript , made by Microsoft. It adds useful language features to Javascript such as a type system , generics , classes and more. The addition of these features helps the code base be more readable and more maintainable.

In particular having types allows the code to be more easily refactored and reduces redundant documentation (such as including type information in comments).

It is compiled to Javascript before being served to the user and all the functionality is compile-time only so there is no performance penalty in using it.

\paragraph{User interface}
The fronted of the application is mostly tailored , in other words the UI components are made specifically for this project. One of the reasons for this is because lecture colour is an important visual cue in the application's visual design and colouring the components of existing platforms such as Bootstrap can be glitchy as they were not meant to be used in such a way.

\paragraph{ReactJS} was used to create the UI components. React is a light javascript library used to create reusable components in a functional way. Using it has helped create powerful , contained reusable components that can be customised externally without needing to dig into their implementation ( which can be useful for future development ) . They also allow for simple colouring by passing around the lecture colour when creating a component.

\subsection{Backend} 
The main backend written in Ruby , using the Ruby on Rails web framework. The main reason for choosing this framework was that it is proven to be reliable and has a very healthy developer ecosystem. Thus there are a lot of up to date libraries ( or 'gems' ) available to extend the core functionality. In particular , the Typescript gem works seamlessly as part of the Rails asset pipeline to convert the typescript files to javascript.

Another backend was written for handling the execution of user code inside docker contains. This was done using Python and Flask , a python web micro framework. A micro framework was used instead of rails because this part of the system does not involve any advanced features found in full web frameworks ( authentication , models , migration for example )

On the same server as the python backend , Docker is used to create instances where the user code can run safely.

\section{Development Tools}
The backend was developed using RubyMine and PyCharm.
The front-end was written using Visual Studio Code , mainly because it was the only editor at the type that supported TSX files ( Typescript + React inline ).


\section{Rails backend analysis}
This section will explain the structure of code within the backend responsible for the main functionality.

\subsection{Routes}
See appendix B

\subsection{Controllers}
Controllers are split by page that the user sees , as well as an additional ApiController which is engineered to be decoupled from the HTML such that it could power a potential mobile or native client.

\paragraph{Application Controller}  Has an action \textit{index} that renders the index page
\paragraph{Lecture Creator Controller} Has an action \textit{index} that renders the page where the teacher can create a lecture
\paragraph{Lecture List Controller} Has an action \textit{index} that renders the page where all the lectures the user created and subscribed to are listed , as well as a \textit{subscribe} method that handles rendering the subscriptions page
\paragraph{Sheet Creator Controller} Has an action \textit{index} that renders the page where the teacher can create sheets
\paragraph{Sheet Dashboard Controller} Has an action \textit{index} that renders the dashboard page that the teacher sees
\paragraph{Sheet Editor Controller} Has an action \textit{index} that renders the sheet editing page where the student can answer a sheet. It also contains an action \textit{update\textunderscore sheet}  that handles updating the user's answer
\paragraph{Sheet Manager Controller} Has an action \textit{index} that renders the page where the teacher can manage all the sheets of his lecture
\paragraph{API Controller} Contains the following methods:
\begin{itemize}
	\item \textit{\textbf{subscribe}} Subscribes the current user to a lecture with id \textit{lecture \textunderscore id}
	\item \textit{\textbf{statistics\textunderscore for \textunderscore question}} Collates and returns the statistics for a given question
	\item \textit{\textbf{completions}} Returns the answers who correctly solved a question
	
	\item \textit{\textbf{lectures}} Returns an object with two properties , \textit{subscribed} which contains the lectures the user is subscribed to and \textit{created} which the user created (is a teacher of)
	\item \textit{\textbf{lecture}} Returns a single lecture information given an id
	\item \textit{\textbf{full \textunderscore sheet}} Returns information about a sheet. Object returned contains the following field : \textit{lecture , sheet , questions , answers , modelAnswers } and \textit{percentage}
	\item \textit{\textbf{sheets}} Returns a list of sheets
	\item \textit{\textbf{create  \textunderscore  sheet}} Creates a sheet given a sheet and an object which represents the new sheet. It will return any errors if any
	\item \textit{\textbf{update \textunderscore sheet}} Updates the name , live status or result status of a given sheet
	\item \textit{\textbf{create \textunderscore lecture}} Creates a lecture with a given name and colour
	\item \textit{\textbf{delete \textunderscore sheet}} Deletes a given sheet
	\item \textit{\textbf{delete \textunderscore lecture}} Deletes a given lecture
	\item \textit{\textbf{lecture \textunderscore users  \textunderscore  count}} Return the number of users subscribed to a lecture
	\item \textit{\textbf{update \textunderscore lecture}} Updates the name or colour of a lecture
	\item \textit{\textbf{stats}} Collates statistics and any metadata computed from them (such as transition matrices)
	\item \textit{\textbf{search}} Searches for a keyword and returns all sheets and lectures that contain it
	\item \textit{\textbf{user \textunderscore info}} Return the email of the logged in user
\end{itemize}

\subsection{Statistics collection}
Statistics are generated when the student answers the sheet questions. They are used to capture metadata that cannot be extracted from just the current state of the user's answers.

To abstract their creation away from other parts of the system , the equals operator is overridden and creates the necessary statistics. The relevant code is shown below:

\begin{lstlisting}[breaklines,language=Ruby,frame=single,basicstyle=\ttfamily,
    stringstyle=\color{red},
    keywordstyle=\color{blue}]
def data=(val)
    if self.question.type == 0

      write_attribute :result , {:correct => val == self.question.correct_answer}.to_json

      if self.data == nil or self.data == ""
        #this means that this is their first choice
        Statistic.create :answer => self , :kind => @@stat_first_click , :data => val
      else
        Statistic.create :answer => self ,
                         :kind => @@stat_change ,
                         :data => {"from" => self.data , "to"=>val}.to_json.to_s
      end

    elsif self.question.type == 1
      write_attribute :result ,
                      {:correct => val.match(Regexp.new(self.question.correct_answer)) != nil}.to_json
    end

    write_attribute :data , val
end
\end{lstlisting}

If the question type is one that should generate a statistic , a statistic is created. In the case the question is a choice question and the user moved from an existing answer to another , a transition statistic is created. This records the change in an object with a "from" and "to" field.

After a statistic is generated , it can be acquired by the client for the dashboard by just acquiring the /statistics/ endpoint with GET over http. The object that represents the transitions is computed this way:

\begin{lstlisting}[breaklines,language=Ruby,frame=single,basicstyle=\ttfamily,
    stringstyle=\color{red},
    keywordstyle=\color{blue}]
transitions = Hash.new {|h, k| h[k]= Hash.new{ |h, k| h[k] = 0}}
c.each{
    |transition|
	 transitions[transition['from']][transition['to']] += 1
}
\end{lstlisting}

A nested hash object is created and using the from and to values a 2d structure is represented , thus generating a grid of numbers with headers that can be easily rendered into a table by the client.

\section{Front end analysis}
The front end is mostly composed of React components; React uses a special file type called JSX\cite{jsx} which allows the mixing of javascript and HTML in the same file. JSX files go though a compilation phase to become plain JS files.

In this project , Typescript is used instead of javascript , so the files written in the project are TSX files , which compile down to JSX and then to JS.

\subsection{Entry points}
The combination of Typescript , React and Rails does not have a lot of official support , this is both because of the fact that typescript/react are newer technologies and that rails favours a javascript alternative called Coffeescript. However a gem called typescript-rails\footnote{Typescript rails gem \url{https://github.com/typescript-ruby/typescript-rails}} provides support for including typescript in the asset pipeline so that typescript .ts files are transparently compiled to .js files so they may work in the browser.

There are however issues when dealing with .tsx files , which contain Typescript + HTML code , the gem does not include them in the pipeline as it most likely has not been updated to support this as of yet. To account for issue , each page has a entry.ts file and that file marks .tsx files as dependencies to it such that the typescript compiler may merge them (after converting TSX to standard typescript). The entry.js is references in the HTML that the server renders and it draws the root React component (often a page component as you will see below).

\subsection{React components}
\paragraph{LectureCreatorPage} The state contains a \textit{\textbf{name}} and \textit{\textbf{color}} which are selected by the user and send using POST when the user presses create lecture.
\paragraph{LectureItem} The component takes a lecture object and renders a single Lecture box
\paragraph{LecturePage} Takes two arrays of lecture objects , one for lectures created and one for subscribed lectures , then renders two sets of LectureItem
\paragraph{SubscribePage} Takes a single lecture object and prompts the user to subscribe to it
\paragraph{SheetCreatorPage} Takes a single lecture object ; this page let's the user create a sheet. It has the following state:
\begin{itemize}
	\item \textit{\textbf{items}} An array of questions that the user is creating for this sheet
	\item \textit{\textbf{name}} The name of the sheet
	\item \textit{\textbf{description}} The description of a sheet
	\item \textit{\textbf{errors}} A map , where the key is a number representing the question index and the value is a list of errors for that question
	\item \textit{\textbf{dragging}} Whether the user is dragging an element to create a new question
\end{itemize}
\paragraph{InputCreator} A component for creating an input question. Takes the following parameters:
\begin{itemize}
	\item \textit{\textbf{color}} The color of the lecture
	\item \textit{\textbf{question}} The question this element is editing/representing
	\item \textit{\textbf{key}} Used internally by React
	\item \textit{\textbf{onDelete}} A function called when the question is deleted
	\item \textit{\textbf{errors}} A list of strings , representing errors with this question
\end{itemize}
\paragraph{CodeIO} An object with an input field and output field (both strings)
\paragraph{CodeCreatorIO} A component with an input textbox and an output textbox. Used to represent a test case in the code question. It takes a CodeIO object which contains a Input and Output field , changes are reflected inside this object
\paragraph{CodeCreator} A component for creating a code question. It allows the user to choose the language and create tests which are instances of CodeIO and are represented by CodeCreatorIO. It has the following properties:
\begin{itemize}
	\item \textit{\textbf{color}} The color of the lecture
	\item \textit{\textbf{question}} The code question this component represents
	\item \textit{\textbf{index}} The index of the question
	\item \textit{\textbf{onDelete}} A function called when the question is deleted
	\item \textit{\textbf{errors}} A list of strings , representing errors with this question
\end{itemize}
\paragraph{ChoiceCreatorAnswer} A component that represents an answer for a choice question. Takes an answer object , the lecture color and a onDelete handler.
\paragraph{ChoiceCreator} A component for creating a choice question. It allows the user to create choices for the user to select.. Takes the following parameters:
\begin{itemize}
	\item \textit{\textbf{color}} The color of the lecture
	\item \textit{\textbf{question}} The question this element is editing/representing
	\item \textit{\textbf{key}} Used internally by React
	\item \textit{\textbf{onDelete}} A function called when the question is deleted
	\item \textit{\textbf{errors}} A list of strings , representing errors with this question
\end{itemize}
\paragraph{SheetDashboardPage} The page where the teacher can see the stats of a sheet. Takes a lecture and a list of questions.
\paragraph{InputStats , CodeStats , ChoiceStats} Components that shows the statistics for the specific type of question. Takes a question and the lecture color.
\paragraph{ActivityIndicator} A component that shows a pie chart that represents the user's who have completed a question succesfully. Takes the following parameters:
\begin{itemize}
	\item \textit{\textbf{question}} The question
	\item \textit{\textbf{color}} The color of the lecture
	\item \textit{\textbf{percentage}} The percentage of students who solved the question
	\item \textit{\textbf{correct}} The number of students who solved the question
	\item \textit{\textbf{wrong}} The number of students who did not solve the question
\end{itemize}
\paragraph{SheetEditPage} The page where the student can complete the sheet. Takes the following properties:
\begin{itemize}
	\item \textit{\textbf{questions}} A list of questions in the sheet
	\item \textit{\textbf{shet}} The sheet the user is answering
	\item \textit{\textbf{lecture}} The lecture the sheet belongs to
	\item \textit{\textbf{answers}} The user's answers
	\item \textit{\textbf{modelAnswers}} The model answers for this sheet
	\item \textit{\textbf{percentage}} The percentage of questions that the user solved
\end{itemize}
\paragraph{InputQuestion , CodeQuestion , ChoiceQuestion} Components that show the questions to the user and let them answer it. Each takes a question , an answer , a color , a model answer and a boolean that represents whether the sheet has model answers released.
\paragraph{SheetManagerPage} The page where the teacher can manage the sheets in the lecture.
\paragraph{SheetControl} A component that represents a sheet and let's the teacher release the model answers , delete it or access the dashboard
\paragraph{SheetsPage} The student view of all the sheets , takes a lecture.
\paragraph{SheetItem} A component that represents a sheet that the student can access. Takes a sheet object.

There are also some shared components:

\paragraph{CheckBox} A checkbox , takes a boolean that represents whether it's checked , a color and a onChange handler.
\paragraph{ColorPicker} Let's the user pick select a color. Takes a onPicked handler that is triggered when a color is selected.
\paragraph{Header} A component that sits at the top of the pages , used for navigation. Takes the following properties:
\begin{itemize}
	\item \textit{\textbf{title}} The title of the page , shown on the left
	\item \textit{\textbf{name}} The name of the user logged in
	\item \textit{\textbf{color}} The color of the lecture
	\item \textit{\textbf{onBack}} Handler , triggered when back button is clicked
	\item \textit{\textbf{foreground}} The foreground color of the header
	\item \textit{\textbf{hideBack}} Whether the back button should be shown.
\end{itemize}
It also has the following state:
\begin{itemize}
	\item \textit{\textbf{showMenu}} Whether the user menu should be shown
	\item \textit{\textbf{showSearch}} Whether to show the search panel
	\item \textit{\textbf{searchSheets}} The sheets found by searching
	\item \textit{\textbf{searchLectures}} The lectures found by searching
	\item \textit{\textbf{query}} The search query typed in the search box
\end{itemize}
\paragraph{IDFactory} Used to generate new IDs for React keys
\paragraph{LCButton} A reusable colourable button , takes a piece of text for the content , a color , an onClick function and a style object.
\paragraph{MDPreview} A reusable component for previewing Markdown. Takes a piece of markdown code and renders it.
\paragraph{SearchDropdown} Used to show search results. Takes the following parameters:
\begin{itemize}
	\item \textit{\textbf{shouldShow}} Whether the search panel should be visible
	\item \textit{\textbf{sheets}} The sheets found by search
	\item \textit{\textbf{lectures}} The lectures found by search
	\item \textit{\textbf{color}} The color of the lecture
	\item \textit{\textbf{query}} The text entered in the search bar
\end{itemize}
\paragraph{SegmentedButton} A segmented button control that let's the user select between options. Takes the following parameters:
\begin{itemize}
	\item \textit{\textbf{color}} The color of the lecture
	\item \textit{\textbf{labels}} An array of strings representing the choices the control provides
	\item \textit{\textbf{style}} Additional styling provided externally
	\item \textit{\textbf{onSelected}} A function triggered when the selection changes
	\item \textit{\textbf{selectedIndex}} The currently selected item
	\item \textit{\textbf{itemPadding}} The padding of each item in the control
\end{itemize}
\paragraph{TextArea} A stylised text area
\paragraph{TextBox} A stylised text box

\subsection{Component styling}
A powerful pattern in react components is to generate css at runtime , this enables features such as the dynamic coloring given by the lecture colour. The project uses some share components that are meant to be used in different pages. Often the page needs a slighly different version of the control , so instead of making flags for all posible variations the component exposes a style property.

\begin{lstlisting}[breaklines,language=Typescript,frame=single,basicstyle=\ttfamily,
    stringstyle=\color{red},
    keywordstyle=\color{blue}]
interface LCButtonProps{
    text : string
    color: string
    onClick? : Function
    style? : any
}
\end{lstlisting}

The style can be any object , then when the component is rendered this object is merged with the static style. If the style property contains a new css rule it will get added ; else if the static style already has that rule , the style property one overrides the static one. It uses jQuery to do this:

\begin{lstlisting}[breaklines,language=Typescript,frame=single,basicstyle=\ttfamily,
    stringstyle=\color{red},
    keywordstyle=\color{blue}]
if(this.props.style != undefined){
	$.extend(containerStyle , this.props.style);    
}
\end{lstlisting}

\section{Unit testing}
Unit testing is used to make sure changes and additions to the system did not break or alter any previous functionality. More precisely it has the following advantages:
\begin{itemize}
	\item To make sure existing functionality was not compromised
	\item To ensure that the database settings worked as intended ( defaults , uniqueness etc)
	\item Components of the system all complied with security , such as preventing user's from editing other people's data or gaining access to data they are not allowed to see
\end{itemize}




\section{Third party libraries} \label{libs}

\subsection{jQuery}
Used in a number of places , such as the above example on merging objects. It is also a dependency of other libraries.

\subsection{React}
A javascript library that allows the project to have isolated , reusable components that build up all of the user interface

\subsection{Chromath} 
A javascript library that can manipulate colours. It is used within the dashboard to compute darker shades of the lecture color - this is useful for both the charts ( each section of the chart is a different colour ) and for the transition matrix ( the higher the transition count , the darker the shade ).

	It is statically included in the project.
	
\subsection{CodeMirror}
A javascript library that provides an embeddable code editor. Used to let user's write their own code.

\subsection{Markdown-it}
A javascript library used to render Markdown , in both the title preview on the sheet creation page and the actual sheet that the user sees.

\subsection{Highlight}
A javascript library to highlight code syntax , used internally by Markdown-it
	
\subsection{JQCloud}
A javascript library used to render a word cloud in the dashboard for input questions.

\subsection{Levenshtein-ffi}
A ruby gem that allows fast calculation of Levenshtein distances for answer's in the input question's dashboard.

\subsection{pg}
A ruby gem for Postgres support

\subsection{Devise}
A ruby gem that simplifies authentication

\subsection{ReactRails}
Support for React in Rails applications

\subsection{Typescript Rails}
Automatic compilation of typescript files as part of the rails asset pipeline.

