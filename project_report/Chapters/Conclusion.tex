\chapter{Conclusion}

\section{Thoughts on project}
This project has taught me a number of things , the most important of which I believe to be making decisions about how time is allocated.
When this project began , my main focus was data aggregation and I had planned to do a lot of work regarding engaging the user in novel ways to try to gauge their under understanding of a subject. However it became apparent to me ( after being advised by my supervisor) that data visualisation and extracting information from the aggregated data is equally important and in fact , the data is not of much use in an unrefined form.
Another thing I learned was the importance of proper mocking up and feature freezes as I have rewritten several components of the system multiple times during the course of the project. This was especially time consuming as I have written most of the UI elements from scratch.

I firmly believe that this project has fulfilled its purpose - which is to create a platform for teacher to user interaction. Both the underlying framework and the user interface have been prepared in such a way that they can extended very easily with more question types and more visualisation options.


\section{Future work} \label{future}
There are a number of ways to improve the project , the system is implemented in such a way that adding more functionality can be done seamlessly without breaking or having to alter any major components of the system.

Ideas for improvements:
\begin{itemize}
	\item More question types. For example questions that revolve around web development ( such as asking the user to create a html/css layout from an image) or unix command line tools ( asking the user to complete some command line workflow ). This should be easy to implement as the database schema for questions is very flexible and does not make any assumptions about the type of data a question expresses
	\item Better user management for the teacher , allowing them to kick , filter and analyse student progress. This would involve adding a lot more UI components and possibly other pages
	\item Allowing teacher to edit the title/body of questions after the sheet is created. Easy to implement the technical aspect of it , however it would not fit within any current place in the UI and would require an additional page
	\item Password enabled lectures , or alternatively geo-locked lectures. Fairly simple to implement but needs to have extra logic in place to deal with changing the password and so on
	\item Mobile apps for Android and iOS to expand the pool of students who have a device that can use the project. Takes time but overall the backend would not have to be changed much as it already exposes most functionality via a simple REST API
\end{itemize}
