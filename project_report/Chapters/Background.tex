\chapter{Background}
\subsection{Relevant concepts}
\paragraph{Question}
Currently there are 3 types of questions:
\begin{itemize}
\item Multiple choice: The user is presented some choice and selects one
\item Input: The user may input any piece of text, multiple solutions can exist
\item Code: Users can write code to solve a posed question
\end{itemize}

The system is built in such a way that adding further questions is trivial and does not require breaking backwards compatibility or redesigning data structures used on the server.

\paragraph{Sheet}
The application revolves around Sheets. Analogue to a paper sheet handed in class, a sheet is a collection of questions of various types. Each sheet belongs to a lecture which has a teacher.


A teacher can create sheets by mix and matching any number of question types which enable quite a varied way to test their students.

\paragraph{Lecture}
A lecture is a collection of sheets. It has a teacher and students can subscribe to a lecture so they can have access to all the sheets.
Dashboard
An important part of this project is for the teacher to be able to monitor the student progress so that they may respond accordingly (for example, write some hints on the board). Each sheet has a dashboard which only the teacher can access.
For each type of question, there is a different specialised user interface that is tailored to convey the progress of the students.
A code question's dashboard widget will show the percentage of students who completed the question. 
An input question’s dashboard shows the percentage of completions, as well as a word cloud of popular words in the questions, this is a specialised control that visually conveys to the teacher common words that are being used.
A choice question’s dashboard will show the user’s completion, a bar chart for top first choices (which could help identify misleading questions amongst other things) and a transition matrix table, which shows the way students move from one answer to another.

The dashboard is also meant to be extensible, so any future question types could provide their own way of visualising student progress.

\subsection{Differentiation from competitors}
The project is meant to be an extensible platform that can be enriched with more types of question along the development process to increase the variety of sheets the teachers can create. This gives it more potential than competitor’s who over inflexible solutions that are tailor made for very specific types of questions.
Furthermore, by focusing on computer science students we allow users to run arbitrary code on the platform, which while some websites allow that – they provide that in a different setting such as code competitions – which means they do not compete with us , as this project enabled the teacher to write a question that can be answered with code as part of a sheet along with other questions.
