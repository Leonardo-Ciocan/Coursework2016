%%%%%%%%%%%%%%%%%%%%%%%%%%%%%%%%%
% 6CCS3PRJ Final Year Individual Project Report
% luke.day@kcl.ac.uk
%%%%%%%%%%%%%%%%%%%%%%%%%%%%%%%%%
\documentclass[11pt]{informatics-report}
\usepackage{color}
\usepackage[square,sort,comma,numbers]{natbib} %References
\usepackage{float}% If comment this, figure moves to Page 2
\usepackage{graphicx}

%%%%%%%%%%%%%%%%%%%%%%%%%%%%%%%%%
% Front Matter - project title, name, supervisor name and date
%%%%%%%%%%%%%%%%%%%%%%%%%%%%%%%%%
\title{6CCS3PRJ Final Year\\\vspace{0.2cm}Orchestrator}
\author{Leonardo Ciocan}
\studentID{1308123}
\supervisor{Jeroen Keppens}

\date{\today}

\abstractFile{FrontMatter/abstract.tex}
\ackFile{FrontMatter/acknowledgements.tex} %Remove line if you do not want acknowledgements

\begin{document}
\createFrontMatter
\doublespacing
\tableofcontents
\doublespacing

%%%%%%%%%%%%%%%%%%%%%%%%%%%%%%%%%
% Report Content
%%%%%%%%%%%%%%%%%%%%%%%%%%%%%%%%%
% You can write each chapter directly here or in a separate .tex file and use the include command.

\chapter{Introduction}
The rise of technology such as mobile devices has been sudden and unexpected , while many industries have quickly adapted to the changes education is one field where embracing technology is still underway. One of the most striking examples of how technology can improve education is the invention of the printer , which enabled literacy to go from 10\% to 80\% in a matter of 50 years \cite{printing}.

A number of other studies show that introducing technology in a learning environment can have a positive impact \cite{tech} and that there is a international movement to augment learning with technology \cite{tech2}.
Today , we live in a world where every student in developed countries has access to a electronic device - in fact , for subjects such as engineering and computer science , virtually every student has access to such a device \cite{laptop1} \cite{laptop2} . In UK universities in particular , paper handouts are still used in lectures ; their purpose is for the students to be able to test how well they understand the current concept being taught and for teachers to gauge the student's understanding of the subject and adjust their teaching schedule accordingly.

A digital replacement of these handouts has the potential to increase student engagement , minimise the friction of distributing , collecting and analysing data and allow the overall turnaround time for the teaching process to be as small as possible.

\section{Motivation}
Handouts can ask a variety of questions ; they can have images , tables , formatted text and more. If a digital alternative is to replace this , it must be flexible enough to support a variety of question types. Further more , paper handouts allow the user to draw or express their answer in a number of ways  , this can be a disadvantage since the students could write invalid answers ( i.e select two answers in a single answer multiple choice question) however it also provides a lot of flexibility. A digital solution must solve both problems by providing flexible , tailored ways to express the questions and to answer them in a way that is valid and yet allow the user to express their creativity.

To account for this goal , the end product build will be tailored for computer science students. This is to showcase the features of the platform and allow for a focused development cycle that would prove the usefulness of a digital solution for this problem. Part of this tailoring will be a question type that let's students write their own code and have it executed and checked by the system.

The data visualisation part of this project provides a number of ways to explore novel ways to show the aggregated data in ways that is useful to the teacher and is tailored for each question type.

In section \ref{future} (Future work) it will be discussed how the system , once shown to work as intended , can be extended further and be tailored for all the subjects that make up higher education. This is because the project will be developed as a platform and make it so adding question types will be simple.


\section{Existing solutions}
There are some products that have attempted to tackle the teacher - student interaction in the classroom.
One of the most popular is called Socrative \footnote{Socrative \url{http://www.socrative.com/}} , which allows teachers to ask the students quizzes. However the system is very limited and only allows text questions , where the answers can be a choice or piece of text. This is not flexible and rich enough to replace traditional methods , only to agreement them. However , it has been downloaded over 1000000 times just on android \footnote{Socrative on Google Play \url{https://play.google.com/store/apps/details?id=com.socrative.student}} , which shows there is a clear market for such solutions.

Another service that could be potentially a competitor is Moodle \footnote{Moodle \url{https://moodle.org/}} . It is not usually used in classroom settings but it does provide support for making quizzes that the students can answer and the teacher can analyse the questions. However this feature is a minor part of Moodle and it does not provide any of the real time advantages of other solutions , including the one described in this project. Some teachers use this to collect code written by user's and then script a solution to execute and mark the answers - however this is not a good solutions as students get little feedback and the process is not automated.

Overall , one of the goals of this project is to bring together the advantages of all those different solutions in a package that is extensible and yet retains the ability to be easy to use for both students and teachers.

\subsection{Platform}
I have decided to build this project for the web instead of a mobile application for a number of reasons. Since this is meant to be used by a lot of students, it is a requirement that it should be as accessible as possible. Statistically speaking, a web app can reach the most people, especially account for the fact that computing students are the main target audience and they are likely to have laptops with them.
The main disadvantage of making mobile apps is that to build a good native app would mean to focus on one platform (as cross platform solutions are not up to the task for the scope of the project) which would exclude students from the process. The web is a free, universal platform and so it is the perfect medium for this project.
Ideally, fully native mobile apps are further down the roadmap as they provide a better experience and further expand the pool of users that are able to use the application.
The backend part of this project has been built in such a way to allow for easy expansion to other platforms. In fact, the web app could be considered just a consumer of the backend and not being renderer by the server such that another front-end could be used instead without any friction.



\subsection{Objectives}
Because the main target audience is made up of computer savvy student and lecturers, it was possible to take a few liberties with the platform. For example the questions can be written in Markdown, which is something computer scientists are already familiar with from websites such as StackOverflow and Github.
Nevertheless, I aim to make a project that requires minimal guidance as the user interface will be built to be easy and intuitive to operate.

The basic requirements for this project are:
\begin{enumerate}
\item Let teacher create sheets made of a variety of question types
\item 	Allow question types to include rich formatting such as tables and code
\item 	Allow students to subscribe to a lecture such that they may have access to those sheets
\item 	Allow a teacher to create a class and invite students with a link
\item 	Let the teacher decide when a sheet should go “live” and be visible by the students
\item 	Allow the teacher to attach model answers to each question
\item Let the teacher release model answers to the students so they may review their answers
\item 	Monitor the students progress and present it to the teacher in an anonymous way
Allow users to write code to be executed for code questions

\end{enumerate}
\subsection{Report structure}
The next section will be the background which will begin by exploring the keywords or concepts that are related to the project.
It will also analyse and contrast current implementation that attempt to solve the problem described with the project's implementation.

\chapter{Background}
\section{Relevant concepts}

\subsection{Teacher}
The word teacher is used throughout the project and its associated materials. It does not necessarily refer to an accredited teacher ; any user of the system can create their own lectures while subscribing to others. Generally speaking , a teacher is one who created a lecture. As a concrete example , a TA\footnote{Teaching assistant} may also benefit from using this in labs.

\subsection{Student}
A student is one who joins a lecture and completes handouts. A user of this system may be both a teacher and a student (of different lectures).

\subsection{Question}
Questions are the core component of this project since handouts are composed of questions. In the context of this project a question is not only a piece of text - a question is composed of a body (the text of the question itself) as well as a type which defines the structure of the answer. Currently there are 3 types of question:

\begin{itemize}
\item Multiple choice: The user is presented some choice and selects one
\item Input: The user may input any piece of text, multiple solutions can exist
\item Code: Users can write code to solve a posed question , the user's code can be run against an arbitrary amount of test cases to confirm it works for a range of inputs
\end{itemize}

The system is built in such a way that adding further questions is trivial and does not require breaking backwards compatibility or redesigning data structures used on the server.

Each question also has a correct answer and a model answer. The correct answer is an object or piece of text that defines what the corrrect answer for the question is. The structure depends on the question type , for example the choice questions's correct answer is defined as a number (the index of the correct answer) , but for input questions it is a regex.

Likewise the model answer is type dependant ; it is shown to the user after the lecture so they may understand how the question is meant to be solved.


\subsection{Sheet}
The application revolves around sheets. Analogue to a paper sheet handed in class, a sheet is a collection of questions of various types. Each sheet belongs to a lecture , all which have a teacher.


A teacher can create sheets by mix and matching any number of question types which enable quite a varied way to test their students.

\paragraph{Going live} When the teacher creates a sheet , it is initially not "live". In other words , students who are subscribed to the lecture cannot yet see it. The teacher may decide when to show the sheet.

\paragraph{Releasing answers} After a sheet has gone live , the teacher may chose to end the sheet and stop students from submitting answers anymore. In this state the student can see the model answer's created by the teacher when designing the sheet as well as their score for that sheet.

\subsection{Lecture}
A lecture is a collection of sheets. It has a teacher and students can subscribe to a lecture so they can have access to all the sheets that the teacher created for the lecture.

Each lecture has a colour , this colour is used throughout the user interface when the student is interacting with content relating with the lecture. This provides a sense of cohesion and makes the user more aware of the lecture they are currently in.

\subsection{Dashboard}
An important part of this project is for the teacher to be able to monitor the student progress so that they may respond accordingly ; for example, write some hints on the board. Each sheet has a dashboard which only the teacher can access.
For each type of question, there is a different specialised user interface that is tailored to convey the progress of the students for that specific type.

A code question's dashboard widget will show the percentage of students who completed the question. 

An input question's dashboard shows the percentage of completions, as well as a word cloud of popular words in the questions, this is a specialised control that visually conveys to the teacher common words that are being used.

A choice question's dashboard will show the user’s completion, a bar chart for top first choices (which could help identify misleading questions amongst other things) and a transition matrix table, which shows the way students move from one answer to another.

The dashboard is also meant to be extensible, so any future question types could provide their own way of visualising student progress.

\section{Existing solutions}
There are some products that have attempted to tackle the teacher - student interaction in the classroom.
One of the most popular is called Socrative \footnote{Socrative \url{http://www.socrative.com/}} , which allows teachers to ask the students quizzes. However the system is very limited and only allows text questions , where the answers can be a choice or piece of text. This is not flexible and rich enough to replace traditional methods , only to agreement them. However , it has been downloaded over 1000000 times just on android \footnote{Socrative on Google Play \url{https://play.google.com/store/apps/details?id=com.socrative.student}} , which shows there is a clear market for such solutions.

Another service that could be potentially a competitor is Moodle \footnote{Moodle \url{https://moodle.org/}} . It is not usually used in classroom settings but it does provide support for making quizzes that the students can answer and the teacher can analyse the questions. However this feature is a minor part of Moodle and it does not provide any of the real time advantages of other solutions , including the one described in this project. Some teachers use this to collect code written by user's and then script a solution to execute and mark the answers - however this is not a good solutions as students get little feedback and the process is not automated.

Overall , one of the goals of this project is to bring together the advantages of all those different solutions in a package that is extensible and yet retains the ability to be easy to use for both students and teachers.


\section{Platform}
This project is built as a web app instead of a mobile application for a number of reasons. Since this is meant to be used by a lot of students, it is a requirement that it should be as accessible as possible. Statistically speaking, a web app can reach the most people, especially account for the fact that computer science students are the main target audience and they are likely to have laptops with them.

The main disadvantage of making mobile apps is that to build a good native app would mean to focus on one platform (as cross platform solutions are not up to the task for the scope of the project) which would exclude students from the process. The web is a free, universal platform and so it is the perfect starting medium for this project.

Ideally, fully native mobile apps are further down the roadmap as they provide a better experience and further expand the pool of users that are able to use the application , especially as tablets and hybrid devices are increasing in popularity with students\cite{tablets}.

The backend part of this project has been built in such a way to allow for easy expansion to other platforms. In fact, the web app could be considered just a consumer of the backend and not being renderer by the server such that another front-end could be used instead without any friction.






















\chapter{Requirements}
Because the main target audience is made up of computer savvy student and lecturers, it was possible to take a few liberties on the front-end of the platform. For example the questions can be written in Markdown, which is something computer scientists are already familiar with from websites such as StackOverflow and Github.

Nevertheless, I aim to make a project that requires minimal guidance as the user interface will be built to be easy and intuitive to operate.

\section{Functional requirements}
\begin{enumerate}
\item   Let teacher create sheets made of a variety of question types
\item 	Allow question types to include rich formatting such as tables and code
\item 	Allow students to subscribe to a lecture such that they may have access to those sheets
\item 	Allow a teacher to create a class and invite students with a link
\item 	Let the teacher decide when a sheet should go “live” and be visible by the students
\item 	Allow the teacher to attach model answers to each question
\item   Let the teacher release model answers to the students so they may review their answers
\item 	Monitor the students progress and present it to the teacher in an anonymous way
\item   Allow users to write code to be executed for code questions
\end{enumerate}


\section{Non-functional requirements}
\begin{enumerate}
	\item User code must be run in a secure environment so that user data is not at risk
	\item Web app must maintain a reasonable size by using shared components and merging duplicate code
	\item Users should not be able to view or change data that does not belong to them
	\item Allow easy extension of question types
	\item Not store identifying information
\end{enumerate}
\chapter{Design \& Specification}

This section will present an abstract view of how the system works.

\subsection{Use cases}
\begin{figure}[H]
  \centering

	\includegraphics[width=\textwidth,height=\textheight,keepaspectratio]{cases}
	\caption{Use cases for students}
\end{figure}

\begin{figure}[H]
  \centering

	\includegraphics[width=\textwidth,height=\textheight,keepaspectratio]{cases2}
	\caption{Use cases for teachers}
\end{figure}

\subsection{System architecture}
\paragraph{User interface}
Since the user's are assumed to be somewhat technically savvy , the application takes some liberties with having a limited amount of help and information banners since it works similarly to other website the target audience is familiar with.
Overall the application is meant to have a flat design that looks good on any resolution.
Each lecture has a colour chosen by the teacher and shows up throughout the user interface so the user knows which lecture they are looking at.

\subsection{Databases}
There is one database in use which contains all the information relevant to the system , it is located on the same server as the backend ; further in the report you will find some notes on how this can be improved and why.
The database schema is very important since changing it could break parts of the system , it was important to chose a structure that would not drastically change later in the development cycle.
Below is the structure of the tables in the system

\subsubsection{Lecture table}
This table contains all the lectures created

\begin{itemize}
	\item \textit{\textbf{Name}} The name of the lecture , there may be multiple lectures with the same name
	\item  \textit{\textbf{Author}} The teacher who created the lecture , can be any user of the system
	\item  \textit{\textbf{Color}} A Hex , RGB or any valid HTML colour , the current UI allows the teacher to choose a colour from a predefined set. It is used in a number of pages , often used to style all controls on the page - such as buttons or tabs	
\end{itemize}

\subsubsection{Sheet table}
This table contains all the sheets created

\begin{itemize}
	\item  \textit{\textbf{Name}} The name of the sheet
	\item  \textit{\textbf{Lecture ID}} The ID of the lecture this sheet belongs to
	\item  \textit{\textbf{Live}} A boolean that represents whether the students who are subscribed to the lecture can see this sheet ( the teacher can use this to choose when to let the students start completing the sheet)
	\item  \textit{\textbf{Released}} A boolean value that represents if the model answers are released (and the user may no longer change his answers)
\end{itemize}

\subsubsection{Question table}
This table contains all the questions created

\begin{itemize}
	\item  \textit{\textbf{Title}} The body of the question as Markdown formatted text ( stored as plaintext)
	\item  \textit{\textbf{Sheet ID}} The ID of the sheet this question belongs to
	\item  \textit{\textbf{Data}} JSON that provides metadata needed to render the question
	\item  \textit{\textbf{Type}} An integer that defines the type of question
	\item  \textit{\textbf{Correct Answer}} A JSON object that defines metadata needed to compute whether the answer is correct or not	
	\item  \textit{\textbf{Model Answer}} A string that represents an example answer
\end{itemize}

\subsubsection{Answer table}
This table contains all the answers created

\begin{itemize}
	\item  \textit{\textbf{Data}} A plaintext representation of the student's answer
	\item  \textit{\textbf{Question ID}} The question that is answered
	\item  \textit{\textbf{User ID}} The ID of the student who created this answer
	\item  \textit{\textbf{Result}} Plaintext field , used to store the cached computed result of the question.
\end{itemize}

\subsubsection{Statistic table}
This table contains all the statistics created when students answer questions

\begin{itemize}
	\item  \textit{\textbf{Answer ID}} The ID of the answer this statistic is about
	\item  \textit{\textbf{Data}} JSON that contains relevant statistical information about the action
	\item  \textit{\textbf{Kind}} The type of statistic this represents
\end{itemize}



\subsubsection{Subscription table}
This table contains all the subscriptions

\begin{itemize}
	\item  \textit{\textbf{Lecture ID}} The ID of the lecture the user is subscribed to
	\item  \textit{\textbf{User ID}} The ID of the user subscribing to the lecture
\end{itemize}






















\chapter{Implementation}

\section{Languages and Frameworks}
\subsection{Front end} 
Since the front end of the project is a web app , HTML and CSS were used to create the user interface. However for the programming Typescript was used instead of Javascript.

\paragraph{Typescript} is a super set of Javascript , made by Microsoft. It adds useful language features to Javascript such as a type system , generics , classes and more. The addition of these features helps the code base be more readable and more maintainable.

In particular having types allows the code to be more easily refactored and reduces redundant documentation (such as including type information in comments).

It is compiled to Javascript before being served to the user and all the functionality is compile-time only so there is no performance penalty in using it.

\paragraph{User interface}
The fronted of the application is mostly tailored , in other words the UI components are made specifically for this project. One of the reasons for this is because lecture colour is an important visual cue in the application's visual design and colouring the components of existing platforms such as Bootstrap can be glitchy as they were not meant to be used in such a way.

\paragraph{ReactJS} was used to create the UI components. React is a light javascript library used to create reusable components in a functional way. Using it has helped create powerful , contained reusable components that can be customised externally without needing to dig into their implementation ( which can be useful for future development ) . They also allow for simple colouring by passing around the lecture colour when creating a component.

\subsection{Backend} 
The main backend written in Ruby , using the Ruby on Rails web framework. The main reason for choosing this framework was that it is proven to be reliable and has a very healthy developer ecosystem. Thus there are a lot of up to date libraries ( or 'gems' ) available to extend the core functionality. In particular , the Typescript gem works seamlessly as part of the Rails asset pipeline to convert the typescript files to javascript.

Another backend was written for handling the execution of user code inside docker contains. This was done using Python and Flask , a python web micro framework. A micro framework was used instead of rails because this part of the system does not involve any advanced features found in full web frameworks ( authentication , models , migration for example )

On the same server as the python backend , Docker is used to create instances where the user code can run safely.

\section{Development Tools}
The backend was developed using RubyMine and PyCharm.
The front-end was written using Visual Studio Code , mainly because it was the only editor at the type that supported TSX files ( Typescript + React inline ).


\section{Rails backend analysis}
This section will explain the structure of code within the backend responsible for the main functionality.

\subsection{Controllers}
Controllers are split by page that the user sees , as well as an additional ApiController which is engineered to be decoupled from the HTML such that it could power a potential mobile or native client.

\paragraph{Application Controller}  Has an action \textit{index} that renders the index page
\paragraph{Lecture Creator Controller} Has an action \textit{index} that renders the page where the teacher can create a lecture
\paragraph{Lecture List Controller} Has an action \textit{index} that renders the page where all the lectures the user created and subscribed to are listed , as well as a \textit{subscribe} method that handles rendering the subscriptions page
\paragraph{Sheet Creator Controller} Has an action \textit{index} that renders the page where the teacher can create sheets
\paragraph{Sheet Dashboard Controller} Has an action \textit{index} that renders the dashboard page that the teacher sees
\paragraph{Sheet Editor Controller} Has an action \textit{index} that renders the sheet editing page where the student can answer a sheet. It also contains an action \textit{update\textunderscore sheet}  that handles updating the user's answer
\paragraph{Sheet Manager Controller} Has an action \textit{index} that renders the page where the teacher can manage all the sheets of his lecture
\paragraph{API Controller} Contains the following methods:
\begin{itemize}
	\item \textit{\textbf{subscribe}} Subscribes the current user to a lecture with id \textit{lecture \textunderscore id}
	\item \textit{\textbf{statistics\textunderscore for \textunderscore question}} Collates and returns the statistics for a given question
	\item \textit{\textbf{completions}} Returns the answers who correctly solved a question
	
	\item \textit{\textbf{lectures}} Returns an object with two properties , \textit{subscribed} which contains the lectures the user is subscribed to and \textit{created} which the user created (is a teacher of)
	\item \textit{\textbf{lecture}} Returns a single lecture information given an id
	\item \textit{\textbf{full \textunderscore sheet}} Returns information about a sheet. Object returned contains the following field : \textit{lecture , sheet , questions , answers , modelAnswers } and \textit{percentage}
	\item \textit{\textbf{sheets}} Returns a list of sheets
	\item \textit{\textbf{create  \textunderscore  sheet}} Creates a sheet given a sheet and an object which represents the new sheet. It will return any errors if any
	\item \textit{\textbf{update \textunderscore sheet}} Updates the name , live status or result status of a given sheet
	\item \textit{\textbf{create \textunderscore lecture}} Creates a lecture with a given name and colour
	\item \textit{\textbf{delete \textunderscore sheet}} Deletes a given sheet
	\item \textit{\textbf{delete \textunderscore lecture}} Deletes a given lecture
	\item \textit{\textbf{lecture \textunderscore users  \textunderscore  count}} Return the number of users subscribed to a lecture
	\item \textit{\textbf{update \textunderscore lecture}} Updates the name or colour of a lecture
	\item \textit{\textbf{stats}} Collates statistics and any metadata computed from them (such as transition matrices)
	\item \textit{\textbf{search}} Searches for a keyword and returns all sheets and lectures that contain it
	\item \textit{\textbf{user \textunderscore info}} Return the email of the logged in user
\end{itemize}



\textcolor{red}{//example code here and in the typescript}

\section{Unit testing}
Unit testing was used to make sure changes and additions to the system did not break or alter any previous functionality.

Purpose:
\begin{enumerate}
	\item To make sure existing functionality was not compromised
	\item To ensure that the database settings worked as intended ( defaults , uniqueness etc)
	\item Components of the system all complied with security , such as preventing user's from editing other people's date or gaining access to data they are not allowed to see
\end{enumerate}




\section{Third party libraries} \label{libs}

\subsection{React}
A javascript library that allows the project to have isolated , reusable components that build up all of the user interface

\subsection{Chromath} 
A javascript library that can manipulate colours. It is used within the dashboard to compute darker shades of the lecture color - this is useful for both the charts ( each section of the chart is a different colour ) and for the transition matrix ( the higher the transition count , the darker the shade ).

	It is statically included in the project.
	
\subsection{CodeMirror}
A javascript library that provides an embeddable code editor. Used to let user's write their own code.

\subsection{Markdown-it}
A javascript library used to render Markdown , in both the title preview on the sheet creation page and the actual sheet that the user sees.

\subsection{Highlight}
A javascript library to highlight code syntax , used internally by Markdown-it
	
\subsection{JQCloud}
A javascript library used to render a word cloud in the dashboard for input questions.

\subsection{Levenshtein-ffi}
A ruby gem that allows fast calculation of Levenshtein distances for answer's in the input question's dashboard.

\subsection{pg}
A ruby gem for Postgres support

\subsection{Devise}
A ruby gem that simplifies authentication

\subsection{ReactRails}
Support for React in Rails applications

\subsection{Typescript Rails}
Automatic compilation of typescript files as part of the rails asset pipeline.


\chapter{Professional Issues}

This project and all of its components abide by the Code of Conduct as detailed by the British Computer Society \cite{conduct} for legal and ethical reasons.

The usage of open source code has greatly accelerated the development of this project and has provided functionality that was outside my means to create in the time frame allocated to this project. As such , every library and instances of code used in this project that were not written by me are clearly stated. Code that is not marked as such can be assumed to written completely for this project.

Since this project revolves around real people in a collaborative setting it was paramount that extra care was put into ensuring the privacy and anonymity of the end users were respected and built into the platform. Minimal user collection is done as the application contains no ads and does not profit from selling the user's information. There are also information barriers to prevent malicious users from exploiting other user's information.
\chapter{Results/Evaluation}

In this section we will explore whether the project has met the requirements that were set in the requirements section of this report as well as other general aspects of the software that can be evaluated.

One of the core requirements was to let the teacher create different types of question and collate them into a sheet. Since this was a core mechanic of the system , it was one of the first things to be implemented. The teacher can create choice questions (student selects one of multiple choices) , input questions (students can write any text into a text box) and code questions (students can write their own code to solve a question. Furthermore the platform was built with this requirement in mind such that it can be extended later.

Markdown is used to write the body of the questions , this allows the teacher to include rich formatting within the question text. This useful because it can make questions more expressive and easier for the user to understand.


Teachers can easily create a lecture and invite students to subscribe to it by sharing a link. After they are subscribed they can access all the sheets that the teacher has chosen to go live. As student complete the sheet , the teacher can monitor the results so that they may know how to proceed further with the lecture. 
When the teacher decides the sheet time is over , he may release the model answers so the user can see their score and what the correct answer is. If the student accesses a sheet after the model answers are released , they will not be able to submit any changes , but may review their answers.

Docker and container technology allow the system to run student code in a safe way. That coupled with the fact that the languages do not have pointer arithmetic / unsafe operations and run in a non root mode ensures the risk of malicious attack using the student code as a vector is minimised if not completely avoided.

The front end has a shared folder which contains components that can be reused across different pages ; they are generic components but can be customised without the need to modify their implementation.

Another requirement was the extensibility of the platform ; the code and database schema are made in such a way that adding more question types and their respective view in the dashboard is trivial. By using JSON for the core definitions of the questions and answers allows the platform to be well positioned for further development.

Anonymity is an important part of this platform , in fact the only identifiable information asked of the user is their email address. Furthermore data is partitioned such that a user may not get access to data they are not supposed to see , they may also not update or attempt to modify or create a resource that belongs to a resource they do not have permission to interact with.

\subsection{Limitations}
There were some things that were simplified or postponed for the sake of keeping in line with deadlines and achieving all the compulsory requirements that I have set. Thus the system does have some limitation (solutions to those limitations will be explored in the next section)

\begin{itemize}
	\item Code question can be either Java or Python based. This is a limitation imposed to decrease the surface area for bugs that would arise from allowing more languages.
	\item Sheets cannot be edited after they are created. This is because changing the questions as the sheet is live could invalidate existing statistical data retrieved from users completing the question.
	\item Teacher cannot kick out or manage users. This is because currently the system aims to anonymise users so they may not feel discouraged from completing the questions
	\item The website is glitchy and may not work at all on some mobile devices. This is especially true for pages where the user may enter a core
\end{itemize}

\subsection{Security}
While the system is not meant to contain any sensitive data , it is important that the users of the system can be assured that their data is private. 
There are safeguards in place to ensure that a malicious agent may not successfully acquire information they are not entitled to , this functionality is unit tested to ensure it stays this way across releases of the software.
Another layer of security is in the system that executes the student's arbitrary code.
The code runs in a docker container and can only run within either the java virtual machine or the python interpreter. Not allowing low level code to run discourages any traditional security holes such as buffer overflows.
Furthermore , even if someone manages to execute code such that they break out of the language runtime and the docker container - the containers run on a server separate from the main server. Thus they cannot access the database or compromise the system.

\subsection{Overall}
The main purpose of the system was to provide teacher's a way to handle handing out sheets digitally and analyse the result easily in real time. It was built in such a way that not only fulfilled this requirement but also provides a platform that is easily extendable such that more question types can be added without having to modify the existing infrastructure.

\chapter{Conclusion and Future Work}

\subsubsection{Future work}
There are a number of ways to improve the project , the system is implemented in such a way that adding more functionality can be done seamlessly without breaking or having to alter any major components of the system.

Ideas for improvements:
\begin{itemize}
	\item More question types. For example questions that revolve around web development ( such as asking the user to create a html/css layout from an image) or unix command line tools ( asking the user to complete some command line workflow ). This should be easy to implement as the database schema for questions is very flexible and does not make any assumptions about the type of data a question expresses
	\item Better user management for the teacher , allowing them to kick , filter and analyse student progress. This would involve adding a lot more UI components and possibly other pages
	\item Allowing teacher to edit the title/body of questions after the sheet is created. Easy to implement the technical aspect of it , however it would not fit within any current place in the UI and would require an additional page
	\item Password enabled lectures , or alternatively geo-locked lectures. Fairly simple to implement but needs to have extra logic in place to deal with changing the password and so on
	\item Mobile apps for Android and iOS to expand the pool of students who have a device that can use the project. Takes time but overall the backend would not have to be changed much as it already exposes most functionality via a simple REST API
\end{itemize}


%%%%%%%%%%%%%%%%%%%%%%%%%%%%%%%%%
% References
%%%%%%%%%%%%%%%%%%%%%%%%%%%%%%%%%
\bibliographystyle{plain}
\bibliography{references}
\addcontentsline{toc}{section}{Bibliography}

%%%%%%%%%%%%%%%%%%%%%%%%%%%%%%%%%
% Appendices
%%%%%%%%%%%%%%%%%%%%%%%%%%%%%%%%%
\appendix
\include{Appendices/appendix}
\chapter{User Guide}
\section{Instructions}
This appendix will contain a short guide to using the product , but note that the product is designed to be intuitive enough not to require a guide.

\paragraph{Accessing the web app} You can either run the project (after installing dependencies) or use the test server at \url{http://178.62.36.214/lectures/}

\paragraph{Logging in} If you are not logged in you will be prompted to either login or signup. Afterwards you'll be redirected to the lectures page.

\paragraph{Creating a lecture} On the lectures page , press the + button next to the label that says "Your own lectures" , enter a lecture name and select a colour , then press "Create"

\paragraph{Creating a sheet} Click your lecture then click "New sheet" , this will take you to the sheet creator page - just fill in the questions you want to make and press "Create Sheet"

\paragraph{Going live} At this point the sheet is not yet visible to the students , press the "Live" button on the sheet card to let students see it

\paragraph{Monitor progress of sheet} To monitor students , press the "Dashboard" button on your sheet card.
\include{Appendices/SourceCode}
\end{document}
