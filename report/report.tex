\title{Final year project}
\author{
        Leonardo Ciocan
}
\date{\today}

\documentclass[12pt]{article}
\usepackage{titlesec}
\titleformat*{\section}{\LARGE\bfseries}

\begin{document}
\maketitle

\clearpage

\begin{abstract}
In university settings, lecturers often teach classes of up to 200 students. Often they distribute papers to gauge the student’s understanding of the current material being taught. The distribution, collection and analysis of these materials makes it hard for the lecturer to evaluate which parts of the material the students need help with.
I have created a digital solution that helps the teacher get a better understanding of his students progress in real time.

\end{abstract}
\clearpage




\section{Introduction}
\subsection{Summary}
There is quite a sizeable market that would benefit from this application, there are 23729 universities around the world; a high percentage of them have Computer Science departments.
Competitors
Some companies have products that aim to facilitate teacher – student interaction but I believe they are rather incomplete and further more none appear to provide tools specifically for computer science students. In other words, they provide generic tools for letting students answer questions but that is not tailor for computer science specifically.
Motivation
The main scope of my project was to create an easily extensible platform that may be useful for teachers specifically teaching computer science as this subject in particular requires some types of question that are more complex than simple text or choices. In other words, letting users write and execute code on my platform was a core goal. Executing arbitrary code is a dangerous so security is an important part of this system.
This project is a great way to experiment with visualisation of data and how digital solutions can enrich previous analogue methods , it also allowed me to work with technologies I had not worked before and build a full product which enriched my knowledge of security , containers , server and backend technologies.

\subsection{Platform}
I have decided to build this project for the web instead of a mobile application for a number of reasons. Since this is meant to be used by a lot of students, it is a requirement that it should be as accessible as possible. Statistically speaking, a web app can reach the most people, especially account for the fact that computing students are the main target audience and they are likely to have laptops with them.
The main disadvantage of making mobile apps is that to build a good native app would mean to focus on one platform (as cross platform solutions are not up to the task for the scope of the project) which would exclude students from the process. The web is a free, universal platform and so it is the perfect medium for this project.
Ideally, fully native mobile apps are further down the roadmap as they provide a better experience and further expand the pool of users that are able to use the application.
The backend part of this project has been built in such a way to allow for easy expansion to other platforms. In fact, the web app could be considered just a consumer of the backend and not being renderer by the server such that another frontend could be used instead without any friction.

\subsection{Objectives}
Because the main target audience is made up of computer savvy student and lecturers, it was possible to take a few liberties with the platform. For example the questions can be written in Markdown, which is something computer scientists are already familiar with from websites such as Stackoverflow and Github.
Nevertheless, I aim to make a project that requires minimal guidance as the user interface will be built to be easy and intuitive to operate.

The basic requirements for this project are:
\begin{enumerate}
\item Let teacher create sheets made of a variety of question types
\item 	Allow question types to include rich formatting such as tables and code
\item 	Allow students to subscribe to a lecture such that they may have access to those sheets
\item 	Allow a teacher to create a class and invite students with a link
\item 	Let the teacher decide when a sheet should go “live” and be visible by the students
\item 	Allow the teacher to attach model answers to each question
\item Let the teacher release model answers to the students so they may review their answers
\item 	Monitor the students progress and present it to the teacher in an anonymous way
Allow users to write code to be executed for code questions

\end{enumerate}

\section{Background}
\subsection{Relevant concepts}
\paragraph{Question}
Currently there are 3 types of questions:
\begin{itemize}
\item Multiple choice: The user is presented some choice and selects one
\item Input: The user may input any piece of text, multiple solutions can exist
\item Code: Users can write code to solve a posed question
\end{itemize}

The system is built in such a way that adding further questions is trivial and does not require breaking backwards compatibility or redesigning data structures used on the server.

\paragraph{Sheet}
The application revolves around Sheets. Analogue to a paper sheet handed in class, a sheet is a collection of questions of various types. Each sheet belongs to a lecture which has a teacher.


A teacher can create sheets by mix and matching any number of question types which enable quite a varied way to test their students.

\paragraph{Lecture}
A lecture is a collection of sheets. It has a teacher and students can subscribe to a lecture so they can have access to all the sheets.
Dashboard
An important part of this project is for the teacher to be able to monitor the student progress so that they may respond accordingly (for example, write some hints on the board). Each sheet has a dashboard which only the teacher can access.
For each type of question, there is a different specialized user interface that is tailored to convey the progress of the students.
A code question's dashboard widget will show the percentage of students who completed the question. 
An input question’s dashboard shows the percentage of completions, as well as a word cloud of popular words in the questions, this is a specialized control that visually conveys to the teacher common words that are being used.
A choice question’s dashboard will show the user’s completion, a bar chart for top first choices (which could help identify misleading questions amongst other things) and a transition matrix table, which shows the way students move from one answer to another.

The dashboard is also meant to be extensible, so any future question types could provide their own way of visualizing student progress.

\subsection{Differentiation from competitors}
The project is meant to be an extensible platform that can be enriched with more types of question along the development process to increase the variety of sheets the teachers can create. This gives it more potential than competitor’s who over inflexible solutions that are tailor made for very specific types of questions.
Furthermore, by focusing on computer science students we allow users to run arbitrary code on the platform, which while some websites allow that – they provide that in a different setting such as code competitions – which means they do not compete with us , as this project enabled the teacher to write a question that can be answered with code as part of a sheet along with other questions.

\subsection{Security}
Because the project allows users to input and run arbitrary code it is important to have proper security in place. There are three layers of security for running code:
\begin{itemize}
\item Users can only run non-native code , currently Python and Java
\item 	All user code runs in secure Docker containers
\item	All containers are contained in a separate server
\end{itemize}
Because the code execution server and the database/main server are separate – even if a malicious agent used a vulnerability to break out of the container, he may not access any information or maliciously disrupt server operations.

//security stats about safe code


\section{Requirements}
\section{Design}
This section will present an abstract view of how the system works.

\subsection{Use cases}
\subsection{System architecture}
\paragraph{User interface}
Since the user's are assumed to be somewhat technically savvy , the application takes some liberties with having a limited amount of help and information banners since it works similarly to other website the target audience is familiar with.
Overall the application is meant to have a flat design that looks good on any resolution.








\end{document}